\rput(2.5,-1.5){\resizebox{!}{4.5cm}{{\epsfbox{images/general/trucos1.eps}}}}
\hypertarget{trucos1}{}\label{trucos}

\pagestyle{trucos}


% -------------------------------------------------
% Cabecera
\begin{flushright}


{\color{introcolor}\mtitle{8cm}{Con un par... de l�neas}}

\msubtitle{5cm}{Los Mejores Trucos}

{\sf por Tamariz el de la Perdiz}

{\psset{linecolor=black,linestyle=dotted}\psline(-12,0)}
\end{flushright}

\vspace{2mm}
% -------------------------------------------------

\begin{multicols}{2}

%\lstset{language=C,frame=tb,framesep=5pt,basicstyle=\footnotesize}
\lstset{language=C,frame=tb,framesep=5pt,basicstyle=\scriptsize}



\hypertarget{documentos-here-y-luxedneas-here}{%
\sectiontext{white}{black}{DOCUMENTOS HERE Y L�NEAS HERE}\label{documentos-here-y-luxedneas-here}}

Probablemente conozcas los documentos \texttt{HERE}. Una forma comoda de
hacer que nuestros scripts escriban varias lineas en pantalla de una
sola vez.

\begin{verbatim}
echo << EOM
Esto son varias lineas que podemos imprimir
Todo lo que escribamos hasta la marca
aparecera en la consola
EOM
\end{verbatim}

Tambien los podemos utilizar para generar ficheros cuando no tenemos un
editor de textos

\begin{verbatim}
cat << EOM  > hola.c
> #include <stdio.h>
> int main(void) { puts("Hola Mundo");}
> EOM
\end{verbatim}

Pero en ocasiones solamente necesitamos una l�nea. Los siguientes
comandos son equivalentes en bash

\begin{verbatim}
$ base64 <<< "Codifica esto en base64"
$ echo "Codifica esto en base64" | base64
\end{verbatim}

Finalmente, la sustituci�n de procesos nos permite utilizar la entrada o
la salida de un comando como entrada o salida de otro. La sustituci�n de
procesos crea un \texttt{pipe} con el que redirigir la entrada o salida
de procesos. Veamos un par de ejemplos

Un ejemplo cl�sico del uso de este operador es la comparaci�n de la
salida de dos procesos. En general, para poder hacer eso, deberiamos
redireccionar la salida de cada proceso a un fichero y luego comparar
esos ficheros. Pero en su lugar podemos hacer lo siguiente.

\begin{verbatim}
$ diff <(cmd1) <(cmd2)
\end{verbatim}

Observad que no hay espacio entre \texttt{\textless{}} y \texttt{(}. De
la misma forma podemos redireccionar la salida de un comando

{\footnotesize
\begin{verbatim}
$ echo "Hola mundo" | tee >(md5sum) >(sha256sum)
\end{verbatim}
}

En este caso el problema esta en que \texttt{tee} espera un fichero como
par�metro, no un comando y por lo tanto no podriamos enciar nuestra
cadena de texto a los dos programas que calculan el hash.

\hypertarget{lista-errores-del-sistema}{%
\sectiontext{white}{black}{LISTA DE ERRORES DEL SISTEMA}\label{lista-errores-del-sistema}}

Durante el desarrollo de programas en C es normal obtener errores como
valores numericos en la variable global \texttt{errno}. Podemos utilizar
la funci�n \texttt{perror} para traducirlos en una cadena comprensible
para nosotros, o podemos utilizar el comando \texttt{errno} para
traducirlos\ldots{} en caso de que, por ejemplo, el programador de
alguna utilidad simplemente nos devuelva un cr�ptico \texttt{error\ 10}

{\small
\begin{verbatim}
$ errno -l  # Muestra todos los errores del sistema
$ errno 10  # Muestra descripcion del error 10
ECHILD 10 No child processes
$ errno EAGAIN # Muestra descripcion del error EAGAIN
EAGAIN 11 Resource temporarily unavailable
\end{verbatim}
}

Si el programa no est� instalado, podr�is encontrarlo en el paquete
\texttt{moreutils} al menos en la distribucion Debian.

\hypertarget{ejecutar-comandos-desde-tu-editor-de-textos-sobre-documento}{%
\sectiontext{white}{black}{EJECUTAR COMANDOS DIRECTAMENTE SOBRE TU TEXTO}\label{ejecutar-comandos-desde-tu-editor-de-textos-sobre-documento}}

Sabias que puedes ejecutar comandos shell sobre bloques de texto que
estas editando en vim. Por ejemplo, puedes ordenar alfabeticamente una
lista que hayas escrito desordenada en tu documento utilizando el
comando:

\begin{verbatim}
:%!cmd
\end{verbatim}

El \texttt{\%} significa aplicar el comando a todo el fichero, pero
tambien podeis usarlo en modo visual.

EN emacs podies hacer lo mismo utilizando el siguiente comando.

\begin{verbatim}
CTRL + U  ALT + | cmd
\end{verbatim}

\hypertarget{servidor-ftp-temporal}{%
\sectiontext{white}{black}{SERVIDOR FTP TEMPORAL}\label{servidor-ftp-temporal}}

Hay muchas formas de transferir ficheros r�pidamente entre ordenadores
de una red. Aqu� os traemos una opci�n m�s, utilizando el modulo
\texttt{pyftplib}. Poner en marcha un servidor web es tan sencillo como
esto.

\begin{verbatim}
$ python -m pyftpdlib --directory=FTP \\
         --port=2121 --write
\end{verbatim}

Para acceder al servidor utilizad el nombre de usuario \texttt{anon}

Pod�is instalar \texttt{pyftplib} con el comando.

\begin{verbatim}
$ pip install --user pyftpdlib
\end{verbatim}

\end{multicols}
